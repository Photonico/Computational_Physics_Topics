\documentclass[a4paper, 12pt]{article}

\usepackage{amsmath}
\usepackage{amssymb}
\usepackage{bm}
\usepackage{enumitem}
\usepackage{fontawesome}
% \usepackage{fontspec}   % with lulatex
\usepackage{geometry}
\usepackage{hyperref}
\usepackage{lipsum}
\usepackage{parskip}
\usepackage{setspace}
\usepackage{tikz}
\usepackage{xcolor}

% \setmainfont{Photonico Code}

% \title{\textbf{A Note for Writing in the Sciences}}
\title{\textbf{A Note for Writing in the Sciences}}
\author{Lu Niu}
\date{\today}

\begin{document}

\maketitle
\vspace{\fill}
\begin{center}
    \par\href{mailto:LukeNiu@outlook.com}{LukeNiu@outlook.com}
    \par\href{http://www.github.com/Photonico}{http://www.github.com/Photonico}
\end{center}
\thispagestyle{empty}

\newpage
This page is intentionally left blank.
\thispagestyle{empty}

\newpage
\thispagestyle{empty}
\tableofcontents
\thispagestyle{empty}

\newpage
\pagenumbering{arabic}

This is my study note for the \textit{Writing in the Science} course,
which is provided through \href{https://www.coursera.org/learn/sciwrite}{Coursera}.

\section*{Prerequisite Knowledge}
\addcontentsline{toc}{section}{Prerequisite Knowledge}

\subsection{Article types}

\vspace{4pt}\textbf{1}. Research Papers: Experimental or theoretical studies on physical phenomena:
\par\quad\textopenbullet\ Physical Review A
\par\quad\textopenbullet\ Physical Review B
\par\quad\textopenbullet\ Physical Review D

\vspace{4pt}\textbf{2}. Review Articles: Summaries of existing research on a physics topic.
\par\quad\textopenbullet\ Physical Review X
\par\quad\textopenbullet\ Reviews of Modern Physics

\vspace{4pt}\textbf{3}. Letters/Short Communications: Brief reports on significant new findings.
\par\quad\textopenbullet\ Physical Review Letter


\vspace{4pt}\textbf{4}. Methods Papers: Descriptions of new experimental techniques or tools.
\par\quad\textopenbullet\ Physical Review Applied
\par\quad\textopenbullet\ Journal of Applied Physics

\vspace{4pt}\textbf{5}. Conference Papers: Short papers for conferences, focusing on recent results.
\par\quad\textopenbullet\ APS March Meeting and APS April Meeting
\par\quad\textopenbullet\ SPIE Conferences

\vspace{4pt}\textbf{6}. Technical Reports: Details on new physics instruments or technology.
\par\quad\textopenbullet\ National Technical Reports Library
\par\quad\textopenbullet\ CERN Document Server
\par\quad\textopenbullet\ NASA Technical Reports Server

\vspace{4pt}\textbf{7}. Commentaries and Perspective Articles: Opinions on current physics issues.
\par\quad\textopenbullet\ Nature Physics
\par\quad\textopenbullet\ Science
\par\quad\textopenbullet\ Physics Today

\par\ \textbullet\ Theses and Dissertations: In-depth studies for degree programs.
\par\ \textbullet\ Textbooks and Educational Materials: Books and materials for physics education.

% Unit 1
\newpage\section{Introduction, principles of effective writing}

\subsection{Introduction}

\par\ \textbullet\ What makes good writing?
\par\quad\textopenbullet\ Good writing communicates an idea clearly and effectively;
\par\quad\textopenbullet\ Good writing is elegant and stylish

\par\ \textbullet\ Clearing writing just takes having something to say and clear thinking.

\par\ \textbullet\ Don't want you to even worry about elegance and stylish when you're writing your first draft, just worry about getting that idea across in a clear and logical, and efficient way.

\par\ \textbullet\ Elegant and stylish writing happens in revision, not on the first draft, even for professional writers. Just barrel through that first draft and then spend time revising.

\par\ \textbullet\ What makes a good writer?
\par\quad\textopenbullet\ Having something to say;
\par\quad\textopenbullet\ Logical thinking;
\par\quad\textopenbullet\ A few simple, learnable rules of style;

\par\ \textbullet\ Become a better writer
\par\quad\textopenbullet\ Read, pay attention, and imitate;
\par\quad\textopenbullet\ Write in a journey;
\par\quad\textopenbullet\ Let go of ``academic'' writing habits (deprogramming step);
\par\quad\textopenbullet\ Talk about your research before trying to write about it;
\par\quad\textopenbullet\ Write to engage your reader: try not to bore them!
\par\quad\textopenbullet\ Stop waiting for ``inspiration'';
\par\quad\textopenbullet\ Accept that writing is hard for everyone, even professional writers;
\par\quad\textopenbullet\ Revise. Nobody gets it perfect on the first try;
\par\quad\textopenbullet\ Learn how to cut ruthlessly, and never become too attached to your words;
\par\quad\textopenbullet\ Find a good editor;
\par\quad\textopenbullet\ Take risks.

\subsection{Examples of what not to do}

\par\ \textbullet\ Ask yourself
\par\quad\textopenbullet\ Is this sentence easy to understand?
\par\quad\textopenbullet\ Is this sentence enjoyable and interesting to read?
\par\quad\textopenbullet\ Is this sentence readable?
\par\quad\textopenbullet\ Is it written to inform or to obscure?

\par\ \textbullet\ The scientific literature should be enjoyable and interesting to read. The point of scientific writing is to inform.

\par\ \textbullet\ The verbs drive sentences whereas nouns slow them down.

\par\ \textbullet\ Themes of this course
\par\quad\textopenbullet\ Complex ideas do not require complex language;
\par\quad\textopenbullet\ Scientific writing should be easy and even enjoyable to read.

\subsection{Overview, principles of effective writing}

\par\ \textbullet\ Avoid the use of acronyms, other than those that are completely standard that most people will know.

\par\ \textbullet\ Do not put too much distance between the subject of the sentence and the main verb.

\par\ \textbullet\ Overview: Principles of effective writing
\par\quad\textopenbullet\ Cut unnecessary words and phrases; learn to part with your words;
\par\quad\textopenbullet\ Use the active voice (subject + verb + object);
\par\quad\textopenbullet\ Write with verbs: use strong verbs, avoid turning verbs into nouns, and don't bury the main verb;

\newpage\subsection{Cut the clutter}

\vspace{4pt}\begin{quote}
The secret of good writing is to strip every sentence to its cleanest components. 
Every word that serves no function, every long word that could be a short word, 
every adverb that carries the same meaning that's already in the verb, 
every passive construction that leaves the reader unsure of who is doing what
- these are thousand and one adulterants that weaken the strength of a sentence. 
And they usually occur in proportion to the education and rank. 
\begin{flushright}
William Zinsser. \textit{On Writing Well}
\end{flushright}
\end{quote}

\par\ \textbullet\ Be vigilant and ruthless

\par\ \textbullet\ After investing much effort to put words on a page, we often find it hard to part with them.

\par\ \textbullet\ Try the sentence without the extra words and see how it's better - conveys the same idea with more power.

\par\textbf{Avoid list}
\par\ \textbullet\ Deadweight words and phrases
\par\quad\textopenbullet\ As it is well known
\par\quad\textopenbullet\ As it has been shown
\par\quad\textopenbullet\ It can be regarded that
\par\quad\textopenbullet\ It should be emphasized that
\par\ \textbullet\ Empty words and phrases
\par\quad\textopenbullet\ basic tenets of
\par\quad\textopenbullet\ methodologic
\par\quad\textopenbullet\ important
\par\ \textbullet\ Long words or phrases that could be short
\par\ \textbullet\ Unnecessary jargon and acronyms
\par\ \textbullet\ Repetitive words or phases
\par\quad\textopenbullet\ studies / examples
\par\quad\textopenbullet\ illustrate / demonstrate
\par\quad\textopenbullet\ challenges / difficulties
\par\quad\textopenbullet\ successful / solutions
\par\ \textbullet\ Adverbs
\par\quad\textopenbullet\ very, really, quite, basically, generally, etc.

\par\ \textbullet\ Some words and phrases are blobs. William Zinsser, On Writing Well.

\par\ \textbullet\ Long words and phrases that could be short
\par\begin{tabular}{lcr}
    \par\textopenbullet\ a number of               & → & many \\
    \par\textopenbullet\ are of the same opinion   & → & agree \\
    \par\textopenbullet\ less frequently occurring & → & rare \\
    \par\textopenbullet\ all three of the...       & → & the three \\
    \par\textopenbullet\ give rise to              & → & cause \\
    \par\textopenbullet\ due to the fact that      & → & because \\
    \par\textopenbullet\ Have an effect on         & → & affect \\
    \par\textopenbullet\ The result of             & → & due to \\
    \par\textopenbullet\ In many instances         & → & often \\
\end{tabular}

\vspace{4pt}\begin{quote}
I have only made this letter rather long because I have not had time to make it shorter.
\begin{flushright}
    Blaise Pascal. \textit{Lettres Provinciales}, 16. Dec.14, 1656.
\end{flushright}
\end{quote}

\subsection{Cut the clutter, more tricks}

\vspace{4pt}\textbf{A few other small tricks}
\par\ \textbullet\ Eliminate negatives
\par\ \textbullet\ Eliminate superfluous uses of ``there are'' / ``there is''
\par\ \textbullet\ Omit needless prepositions
\par\ \textbullet\ Use an exciting verb rather than a ``to be'' verb.

\newpage\subsection{Practicing cutting clutter}

The best way to learn writing is through practice.

\par\textbf{Examples}

\par\ \textbullet\ Before
\par Anti-inflammatory drugs may be protective against the occurrence of Alzheimer’s Disease.

\par\ \textbullet\ After
\par Anti-inflammatory drugs protect against Alzheimer’s Disease.

\par\ \textbullet\ Before
\par Clinical seizures have been estimated to occur in 0.5% to 2.3% of the neonatal population.

\par\ \textbullet\ After
\par Clinical seizures occur in 0.5% to 2.3% of newborns.

\par\ \textbullet\ Before
\par Ultimately p53 guards not only against malignant transformation but also plays a role in developmental processes as diverse as aging, differentiation, and fertility.

\par\ \textbullet\ After
\par Besides preventing cancer, p53 also works in aging, differentiation, and fertility.

\par\ \textbullet\ Before
\par Injuries to the brain and spinal cord have long been known to be among the most devastating and expensive of all injuries to treat medically.

\par\ \textbullet\ After
\par Injuries to the brain and spinal cord are among the most devastating and expensive.

\par\ \textbullet\ Before
\par An IQ test measures an individual's abilities to perform functions that usually fall in the domains of verbal communication, reasoning, and performance on tasks that represent motor and spatial capabilities.

\par\ \textbullet\ After
\par An IQ test measures an individual's verbal, reasoning, or motor and spatial abilities.

\par\ \textbullet\ Before
\par As we can see from Figure 2, if the return kinetic energy is less than 3.2 Up, there will be two electron trajectories associated with this kinetic energy.

\par\ \textbullet\ After
\par Figure 2 shows that a return kinetic energy less than 2.3 Up yields two electron trajectories.

\newpage\section{Writing with strong}

\subsection{Use the active voice}

Use strong verbs, avoid turning verbs into nouns, and keep the subject and main verb close together at the beginning.

\par\textbf{Active voice}
\par\ \textbullet\ Subject \quad→\quad verb \quad→\quad object.

\par\textbf{Passive voice}
\par\ \textbullet\ A form of the verb ``to be'' + the past participle of the main verb;
\par\ \textbullet\ The main verb must be transitive;
\par\ \textbullet\ The passive voice usually takes out the responsible party.

\par\textbf{To turn the passive voice back to the active voice}
\par\ \textbullet\ Who does what to whom? - The agent.

\par\textbf{Advantages of the active voice}
\par\ \textbullet\ Emphasizes author responsibility;
\par\ \textbullet\ Improves readability;
\par\ \textbullet\ Reduces ambiguity.

\par\textbf{Is it ever OK to use the passive voice?}
\par Yes! The passive voice exists in the English language for a reason. Just use it sparingly and purposefully. For example, passive voice may be appropriate in the methods section where what was done is more important than who did it.

\newpage\subsection{Is it OK to use ``We'' and ``I''}

\par\textbf{It's okay to use personal pronouns.}
\par\ \textbullet\ he active voice is livelier and easier to read;
\par\ \textbullet\  Avoiding personal pronouns does not make your science more objective;
\par\ \textbullet\ Journals want this;
\par\ \textbullet\ By agreeing to be an author on the paper, you are taking responsibility for its content. Thus, you should also claim responsibility for assertions in the text by using personal pronouns.


\par\textbf{An example of an opening sentence in a paper}

\vspace{4pt}\begin{quote}
We wish to suggest a structure for the salt of deoxyribose nucleic acid (D.N.A.). This structure has novel features which are of considerable biological interest.
\begin{flushright}
\textit{Nature}, 171, 737-738. April 25, 1953.
\end{flushright}
\end{quote}

\subsection{Active voice practice}

Text of practice exercises for Module 2.3:

\par\ \textbullet\ Before
\par A recommendation was made by the DSMB committee that the study be halted.

\par\ \textbullet\ After
\par The DSMB committee recommended that the study be halted.

\par\ \textbullet\ Before
\par Major differences in the reaction times of the two study subjects were found.

\par\ \textbullet\ After
\par We observed major differences in the reaction times of the two study subjects.

\par\ \textbullet\ Or
\par The two study subjects differed in reaction times.

\par\ \textbullet\ Before
\par It was concluded by the editors that the data had been falsified by the authors.

\newpage\par\ \textbullet\ After
\par The editors concluded that the authors falsified their data.

\par\ \textbullet\ Before
\par The first visible-light snapshot of a planet circling another star has been taken by NASA’s Hubble Space Telescope.

\par\ \textbullet\ After
\par NASA's Hubble Space Telescope has taken the first visible-light snapshot of a planet circling another star.

\par\ \textbullet\ Before
\par Therefore, the hypothesis that the overall kinetics of a double transtibial amputee athlete and an able-bodied sprinter at the same level of performance is not different was rejected.

\par\ \textbullet\ After
\par Therefore, we rejected the hypothesis that the overall kinetics of a double transtibial amputee athlete and an able-bodied sprinter at the same level of performance is comparable.

\par\ \textbullet\ If there is no agent, we can assume it's the authors of that article.

\newpage\subsection{Write with verbs}

\par\textbf{Three principles}
\par\ \textbullet\ Use strong verbs
\par\ \textbullet\ Avoid turning verbs into nouns
\par\ \textbullet\ Do not bury the main verb

\par\textbf{Use strong verbs}
\par\ \textbullet\ Verbs make sentences go;
\par\ \textbullet\ Pick the right verb, avoid the use of adverbs;
\par\ \textbullet\ Use ``to be'' verbs purposefully and sparingly: they are overused and boring,

\par\textbf{Avoid turning verbs into nouns}
\par\ \textbullet\ Examples
\par\begin{tabular}{lcr}
    \par\textopenbullet\ Obtain estimates of & → & estimate \\
    \par\textopenbullet\ Has seen an expansion in & → & has expanded \\
    \par\textopenbullet\ Provides a methodologic emphasis & → & emphasizes methodology \\
    \par\textopenbullet\ Take an assessment of & → & assess \\
    \par\textopenbullet\ Provide a review of & → & review \\
    \par\textopenbullet\ Offer confirmation of & → & confirm \\
    \par\textopenbullet\ Make a decision & → & decide \\
    \par\textopenbullet\ Shows a peak & → & peaks \\
\end{tabular}


\par\textbf{Do not bury the main verb}
\par\ \textbullet\ Keep the subject and main verb: (Predicate) close together at the start of the sentence.
\par\quad\textopenbullet\ Readers are waiting for the verb.

\newpage\subsection{Practice examples}

\par\ \textbullet\ Before
\par The fear expressed by some teachers that students would not learn statistics well if they were permitted to use canned computer programs has not been realized in our experience. Careful monitoring of achievement levels before and after the introduction of computers in the teaching of our course revealed no appreciable change in students’ performances.

\par\ \textbullet\ After
\par Many teachers feared that the use of canned computer programs would prevent students from learning statistics. We monitored student achievement levels before and after the introduction of computers in our course and found detriments in performance.

\par\ \textbullet\ Before
\par Review of each center’s progress in recruitment is important to ensure that the cost involved in maintaining each center’s participation is worthwhile.

\par\ \textbullet\ After
\par We should review each center's recruitment progress to make sure its continued participation is cost-effective.

\par\ \textbullet\ Before
\par It should be emphasized that these proportions generally are not the result of significant increases in moderate and severe injuries, but in many instances reflect mildly injured persons not being seen at a hospital.

\par\ \textbullet\ After
\par Shifting proportions in injury severity may reflect stricter hospital admission criteria rather than true increases in moderate and severe injuries.

\par\ \textbullet\ Before
\par Important studies to examine the descriptive epidemiology of autism, including the prevalence and changes in the characteristics of the population over time, have begun.

\par\ \textbullet\ After
\par Studies have begun to describe the epidemiology of autism, including recent changes in the disorder's prevalence and characteristics.

\par\ \textbullet\ Before
\par There are multiple other mechanisms that are important, but most of them are suspected to only have a small impact or are only important because of the impact on one of the three primary mechanisms.

\par\ \textbullet\ After
\par Multiple other mechanisms play only a small role or work by impacting one of the three primary mechanisms.

\par\ \textbullet\ Before
\par After rejecting paths with poor signal-to-noise ratios, we were left with 678 velocity measurements of waves with 7.5 seconds period and 891 measurements of 15 second waves.

\par\ \textbullet\ After
\par Rejecting path with poor signal-to-noise ratios left 678 velocity measurements of 7.5 second waves and 891 of 15 second waves.

\par\ \textbullet\ Before
\par It is suspected that the importance of temperature has more to do with impacting rates of other reactions than being a mechanism of disinfection itself since ponds are rarely hot enough for temperature alone to cause disinfection.

\par\ \textbullet\ After
\par Ponds are rarely hot enough for temperature alone to cause disinfection; thus, the effect of temperature is likely mediated through its impact on the rates of other reactions.

\par\ \textbullet\ Before
\par It was assumed that due to reduced work at joints of the lower limbs and less energy loss in the prosthetic leg, running with the dedicated prostheses allows for maximum sprinting at lower metabolic costs than in the healthy ankle joint complex.

\par\ \textbullet\ After
\par The prosthetic leg reduced work and energy loss compared with a healthy ankle joint, which may lead to lower metabolic costs during maximum sprinting.

\newpage\subsection{A few grammar tips}

\textbf{The word ``data'' is plural.}
\par\ \textbullet\ These data show an unusual trend.

\textbf{Affect as a verb, denotes ``to influence''.}
\par\ \textbullet\ The class affected her.

\textbf{Effect as a noun, denotes the form of this influence.}
\par\ \textbullet\ The class had an effect on her.

\textbf{Affect as a noun, denotes feeling or emotion shown by facial expression or body language.}
\par\ \textbullet\ The soldiers seen on television had been carefully chosen for blandness of effect.

\textbf{Effect as a verb, denotes to bring about or to cause.}
\par\ \textbullet\ to effect a change.

\textbf{That vs. which}
\par\ \textbullet\ That: the restrictive (defining) pronoun;
\par\quad\textopenbullet\ The vial that contained her RNA was lost.
\par\ \textbullet\ Which: the nonrestrictive (no-defining) pronoun.
\par\quad\textopenbullet\ The vial, which contained her RNA, was lost.

\textbf{Key question: Is your clause essential or non-essential?}
\par\ \textbullet\ That: The essential clause cannot be eliminated without changing the meaning of the sentence;
\par\ \textbullet\ Which: The non-essential clause can be eliminated without altering the basic meaning of the sentence (and must be set off by commas).

\vspace{4pt}\begin{quote}
Care writers, watchful for small conveniences, go which-hunting, remove the defining whiches, and by doing so improve their work.
\begin{flushright}
Strunk and White.
\end{flushright}
\end{quote}

\newpage\par\ \textbullet\ Before
\par Stroke incidence data are obtained from sources, which use the ICD (International Code of Diseases classification system).

\par\ \textbullet\ After
\par Stroke incidence data are obtained from sources that use the ICD (International Code of Diseases classification system).

\par\ \textbullet\ From Physicist Richard Feynman:
\par When we say we are a pile of atoms, we do not mean we are merely a pile of atoms because a pile of atoms, which is not repeated from one to the other might well have the possibilities which you see before you in the mirror.

\par\ \textbullet\ After
\par When we say we are a pile of atoms, we do not mean we are merely a pile of atoms because a pile of atoms that is not repeated from one to the other might well have the possibilities that you see before you in the mirror.

\textbf{Singular antecedents}
\par\ \textbullet\ Do not use ``they'' or ``their'' when the subject is singular.
\par\ \textbullet\ To avoid gender choice, turn to a plural.

\newpage\section{Punctuation}

\subsection{Experiment with punctuation}

Our friends the dash, colon, semicolon, and parenthesis... Use them to vary sentence structure.

\textbf{Increasing power to separate}
\par\ \textbullet\ Comma → Colon → Dash → Parentheses → Semicolon → Period

\textbf{Increasing formality}
\par\ \textbullet\ Dash → Parentheses → The Others (Comma, Colon, Semicolon, Period)

\textbf{Semicolon}
\par\ \textbullet\ The semicolon connects two independent clauses. A clause always contains a subject and predicate; an independent clause can stand alone as a complete sentence.
\par\ \textbullet\ Semicolons are also used to separate items in lists that contain internal punctuation.

\textbf{Parenthesis (parenthetical expression)}
\par\ \textbullet\ Use parentheses to insert an afterthought or explanation (a word, phrase, or sentence) into a passage that is grammatically complete without it;
\par\ \textbullet\ If you remove the material within the parentheses, the main point of the sentence should not change;
\par\ \textbullet\ Parentheses give the reader permission to skip over the material.

\textbf{Colon}
\par\ \textbullet\ Use a colon after an independent clause to introduce a list, quote, explanation, conclusion, or amplification.
\par\ \textbullet\ Use a colon to join two independent clauses if the second amplifies or extends the first.

\textbf{The ``rule of three's'' for lists and examples''}
\par\ \textbullet\ It happened because people organized and voted for better prospects; because leaders enacted smart, forward looking policies; because people's perspectives opened up, and with them, societies did too.

\newpage\textbf{Dash}

\vspace{4pt}\begin{quote}
    A dash is a mark of separation stronger than a comma, less formal than a colon, and more relaxed than paratheses
\begin{flushright}
Strunk and White.
\end{flushright}
\end{quote}

\vspace{4pt}\begin{quote}
    Use a dash only when a more common mark of punctuation seems inadequate.
\begin{flushright}
Strunk and White.
\end{flushright}
\end{quote}

\subsection{Practice, colon, and dash}

\par\ \textbullet\ Before
\par Evidence-based medicine teaches clinicians the practical application of clinical epidemiology, as needed to address specific problems of specific patients. It guides clinicians on how to find the best evidence relevant to a specific problem, how to assess the quality of that evidence, and perhaps most difficult, how to decide if the evidence applies to a specific patient.

\par\ \textbullet\ After
\par Evidence-based medicine teaches clinicians the practical application of clinical epidemiology: how to find the best evidence relevant to a specific problem, how to assess the quality of that evidence, and how to decide if the evidence applies to a specific patient.

\par\ \textbullet\ Or
\par Evidence-based medicine teaches clinicians how to find the best evidence relevant to a specific problem, how to assess the quality of that evidence, and how to decide if the evidence applies to a specific patient.

\newpage\par\ \textbullet\ Before
\par Finally, the lessons of clinical epidemiology are not meant to be limited to academic physician-epidemiologists, who sometimes have more interest in analyzing data than caring for patients. Clinical epidemiology holds the promise of providing clinicians with the tools necessary to improve the outcomes of their patients.

\par\ \textbullet\ After
\par Finally, clinical epidemiology is not limited to academic physician - epidemiologists - who are sometimes more interested in analyzing data than caring for patients - but provides clinicians with the tools to improve their patients' outcomes.

\subsection{Parallelism}

Pairs of ideas joined by ``and'', ``or'', or ``but'' should be written in parallel form.

\textbf{A list of ideas should be written in parallel form}
\par\ \textbullet\ Unparallel
\par Locusts denuded fields in Utah, rural Iowa was washed away by torrents, and in Arizona, the cotton was shriveled by the placing heat.

\par\ \textbullet\ Parallel
\par Locusts denuded fields in Utah, torrents washed away rural Iowa, and blazing head shriveled Arizona's cotton.

\newpage\subsection{Paragraphs}

\textbf{Paragraph-level tips}

\par\ \textbullet\ 1 paragraph = 1 idea
\par\quad\textopenbullet\ Let the reader appreciates short paragraphs and white space on the page;
\par\quad\textopenbullet\ Try to keep your paragraphs short and focused on a single idea.

\par\ \textbullet\ Give away the punch line early.
\par\quad\textopenbullet\ Present the main findings or conclusions at the beginning of a scientific paper.

\par\ \textbullet\ Paragraph flow is helped by:
\par\quad\textopenbullet\ A logical flow of ideas
\par\quad\textopenbullet\ parallel sentence structures
\par\quad\textopenbullet\ if necessary, transition words

\par\ \textbullet\ Your reader remembers the first sentence and the last sentence best. Make the last sentence memorable. Emphasis at the end.

\par\ \textbullet\ A logical flow of ideas:
\par\quad\textopenbullet\ Sequential in time: avoid the ``Memento'' approach in scientific writing;
\par\quad\textopenbullet\ General → specific: take-home message first;
\par\quad\textopenbullet\ Logical arguments: if ``a'' then ``b''; ``a'' therefore ``b''.

\newpage\subsection{Paragraph Editing I}

If you need to keep telling the reader where you are going, this usually indicates problems with the underlying logic.

\par\ \textbullet\ Before
\par Most scents remain constant in their quality over orders of magnitude of concentration (12). 
\textbf{Nevertheless}, at high concentrations, quality tends to be negatively correlated with intensity, as was the case, 
for example, for the cinnamon oil used in this study.
\textbf{Hence}, the reliability of absolute scorings was achieved by calibrating the number of perfume ingredients with initial ratings for intensity against a reference substance of known concentration. The final concentrations were in principle chosen in a way such that individual ratings showed variance among participants within the sliding scale between 0 and 10 (meaning that people could decide whether they liked a scent or not). 
This procedure seemed successful for most scents;
\textbf{however}, the concentrations for bergamot (highest average ratings) and vetiver (lowest average rating) could probably be reduced even more, as both scents did not show any discriminating power at the level of common alleles (people agreed largely on the quality of these two scents) (see Table 2). 
\textbf{Interestingly}, the pooled rare alleles showed discriminating power for...

\par\ \textbullet\ After
\par Perfume intensity and quality are negatively correlated at high concentrations:
If the scent is too strong, people will rate it unfavorably.
\textbf{Hence}, we chose the final concentration of each perfume ingredient so that it had a similar intensity to a preference scent (1-butanol).
The resulting concentrations appeared appropriate for most scents, as participants' preferences varied along the sliding scale between 0 and 10.
\textbf{However}, participants largely agreed on bergamot (highest average rating) and vetiver (lowest average rating),
so lower or higher concentrations may have been needed for these scents.

\newpage\par\ \textbullet\ Before
\par \textbf{Although} the methodological approaches are similar, the questions posed in classic epidemiology and clinical epidemiology are different.
In classic epidemiology, epidemiologists pose a question about the etiology of a disease in a population of people.
Causal associations are important to identify because, if the causal factor identified can be manipulated or modified, prevention of disease is possible.
\textbf{On the other hand}, in clinical epidemiology, clinicians pose a question about the prognosis of a disease in a population of patients. Prognosis can be regarded as a set of outcomes and their associated probabilities following the occurrence of some defining event or diagnosis that can be a symptom, sign, test result or disease.

\par\ \textbullet\ After
\par \textbf{Despite} methodologic similarities, classic epidemiology, and clinical epidemiology differ in aim.
Classic epidemiologists pose a question about the etiology of a disease in a population of people.
Clinical epidemiologists pose a question about the prognosis of a disease in a population of patients;
prognosis is the probability that an event or diagnosis will result in a particular outcome.

\newpage\subsection{Paragraph Editing II}

\par\ \textbullet\ Before
\par The concept of chocolate having potential therapeutic benefits for people with diabetes mellitus, especially type 2 diabetes mellitus, presents a number of intellectual challenges, from both clinical and sociological perspectives. It seems almost counterintuitive to suggest an energy-dense food that is high in sugar, and often seen as a treat or a “dietary sin”, could offer such promise. However, a large volume of mechanistic and animal model studies has been undertaken demonstrating the potential benefits of cocoa and chocolate for both glucose regulation and modification of complications associated with diabetes. Cesar Fraga in the American Journal of Clinical Nutrition first proposed the potential of chocolate for people with diabetes in 2005. It was suggested that we should consume more cocoa and chocolate to reduce the burdens of hypertension and diabetes. [1] Grassi and colleagues (2) further reinforced this potential for its antihypertensive and insulin-sensitizing effect with the mechanistic data. However, the hypothesis of chocolate having a beneficial effect remains counterintuitive to the average consumer and has yet to gain support among the wider medical and healthcare community.

\par\ \textbullet\ After
\par Many mechanistic animal studies suggest health benefits for cocoa and chocolate, particularly for patients with hypertension and type II diabetes mellitus. These studies suggest that cocoa and chocolate can lower blood pressure, improve glucose regulation, improve insulin sensitivity, and reduce complications associated with diabetes. But the idea of chocolate as medicine has yet to gain widespread support among consumers or the wider medical and healthcare community. It seems that high-sugar, energy-dense food - one often seen as a treat or a ``dietary sin'' - could promote health.

\newpage\par\ \textbullet\ Before
\par Headache is an extraordinarily common pain symptom that virtually everyone experiences at one time or another. As a pain symptom, headaches have many causes. The full range of these causes were categorized by the International Headache Society (IHS) in 1988. The IHS distinguishes two broad groups of headache disorders: primary headache disorders and secondary headache disorders. Secondary headache disorders are a consequence of an underlying condition, such as a brain tumor, a systemic infection, or a head injury. In primary headache disorders, the headache disorder is the fundamental problem; it is not symptomatic of another cause. The two most common types of primary headache disorders are episodic tension-type headache (ETTH) and migraine. Although IHS is the most broadly used/recognized classification system used, a brief comment on others would be appropriate – especially if there are uses that have epidemiologic advantages.

\par\ \textbullet\ After
\par Headache is a pain symptom that almost everyone experiences. The International Headache Society (IHS) groups headaches into two types based on cause: primary headache disorders and secondary headache disorders. In primary headache disorders, the headache itself is the main complaint. The two most common types of primary headache disorder are episodic tension-type headache (ETTH) and migraine. Secondary headache disorders result from an underlying condition, such as a brain tumor, a systemic infection, or a head injury.

\newpage\subsection{A few more tips}

When you find yourself reaching for the thesaurus to avoid using a word twice within the same sentence or even paragraph, ask:
\par\ \textbullet\ Is the second instance of the word even necessary?
\par\ \textbullet\ If the word is needed, is a synonym better than just repeating the word?
\par\quad\textopenbullet\ challenges / difficulties
\par\quad\textopenbullet\ illustrate / demonstrate
\par\quad\textopenbullet\ teaches / guides

\textbf{A note on repetition}
\par\ \textbullet\ In many cases, it's preferable to just repeat the word.
\par\ \textbullet\ In scientific writing, you must repeat a word.
\par\ \textbullet\ Repeat the keywords.
\par\quad\textopenbullet\ names of comparison groups, variables, or instruments.

\textbf{Needless synonyms}
\par\begin{tabular}{lcr}
    \par\textopenbullet\ banana  & → & ``the elongated yellow fruit'' \\
    \par\textopenbullet\ beaver  & → & ``the furry, paddle-tailed mammal'' \\
    \par\textopenbullet\ mustache  & → &  ``under-nose hair crops'' \\
    \par\textopenbullet\ milk from a cow  & → & ``the vitamin-laden liquid from a bovine milk factory'' \\
    \par\textopenbullet\ skis  & → & ``the beatified barrel staves'' \\
\end{tabular}

\textbf{Disastrous synonyms}
\par\ \textbullet\ Whereas it's just amusing or inelegant in some types of writing, in scientific writing it's a disaster.
\par\ \textbullet\ The reader may think you are referring to different instruments, models, groups, variables, etc.

\textbf{Acronyms / Initialisms}
\par\ \textbullet\ It is OK to repeat words. Resist the temptation to abbreviate words simply because they recur frequently.
\par\quad\textopenbullet\ Recall miR instead of microRNA
\par\ \textbullet\ Use only standard acronyms/initialisms. Do not make them up.
\par\quad\textopenbullet\ RNA
\par\ \textbullet\ If you must use acronyms, define them separately in the abstract, each table/figure, and the text. For long papers, redefine occasionally (as readers do not typically read from start to finish).

\textbf{Reference}
\par Henry W. Fowler on [Elegant Variation](http://www.bartleby.com/116/302.html)

\newpage\subsection{Exercise}

\par\ \textbullet\ Before
\par As for many food components, the intake of metal ions can be a double-edged sword.
The requirement for ingestion of trace metals such as Fe and Cu ions to maintain normal body functions such as the synthesis of metalloproteins is well established.
However, cases of excess intake of trace metal ions are credited with pathological events such as the deposition of iron oxides in Parkinson's disease [1].
In addition to aiding neurological depositions, these redox active metals ions have been credited with enhancing oxidative damage,
a key component of chronic inflammatory disease [2] and a suggested initiator of cancer [3]. As inflammation is a characteristic feature of a wide range of diseases,
further potential pathological roles for metal ions are emerging as exemplified by premature aging [4].

\par\ \textbullet\ After
\par Metal ions' intake cuts both ways in food components.
Our bodies require trace metals, such as Fe and Cu ions, to maintain normal functions.
However, studies have demonstrated that excessive intake contributes to pathological events,
such as Parkinson's disease [1]; oxidative damage [2]; cancer [3], and premature aging [4].

\newpage\subsection{Assignment I}

\par Revise the following paragraph to improve clarity, brevity, and organization:

\par As for many food components, the intake of metal ions can be a double-edged sword. 
The requirement for ingestion of trace metals such as Fe and Cu ions to maintain normal body functions such as the synthesis of metalloproteins is well established.
However, cases of excess intake of trace metal ions are credited with pathological events such as the deposition of iron oxides in Parkinson's disease [1].
In addition to aiding neurological depositions, these redox active metals ions have been credited with enhancing oxidative damage,
a key component of chronic inflammatory disease [2] and a suggested initiator of cancer [3]. As inflammation is a characteristic feature of a wide range of diseases,
further potential pathological roles for metal ions are emerging as exemplified by premature aging [4].

\par\ \textbullet\ Answer

\par Metal ions' intake cuts both ways in food components. Our bodies require trace metals, such as Fe and Cu ions, to maintain normal functions.
However, studies have demonstrated that excessive intake contributes to pathological events,
such as Parkinson's disease [1]; oxidative damage [2]; cancer [3], and premature aging [4].

\newpage\section{Writing steps}

\subsection{More paragraph practice}

\par\ \textbullet\ Example
\par Syoersymmetry relates each particle of the standard model to another particle called its superpartner;
the symmetry is about spin-a standard-model fermion has a bosonic superpartner, and vice versa.
By convention, superpartners of fermions gain a prefix `s' (such as selectron, squark, and sneutrino), and those related to bosons gain the suffix ``ino''.
The prime candidate for dark matter among all of these superparticles is the so-called neutralino - which is a mixture (technically, a mass eigenstate) formed by the superpartners (zino, photino, and higgsino) of standard-model bosons. Other candidates are sneutrinos, and gravitinos, which are related to the graviton (although strictly speaking, gravitons belong to extended versions of supersymmetric models, known as supergravity models, in which gravity is included.)

\par\ \textbullet\ Before
\par In assessing the quality of an instrument we distinguish three quality domains,
i.e. reliability, validity, and responsiveness. Each domain contains one or more measurement properties.
The domain reliability contains three measurement properties: internal consistency, reliability, and measurement error.
Domain validity also contains three measurement properties: content validity, construct validity, and criterion validity.
The domain responsiveness contains only one measurement property, which is also called responsiveness.
The term and definition of the domain and measurement property responsiveness are actually the same, but they are distinguished in the taxonomy for reasons of clarity.
Some measurement properties contain one or more aspects, that were defined separately: Content validity includes face validity,
and construct validity includes structural validity, hypotheses testing, and cross-cultural validity.

\par\ \textbullet\ After
\par We assess each instrument based on reliability, validity, and responsiveness.
These domains may be subdivided into measurement properties: Reliability includes internal consistency,
reliability, and measurement error; validity includes content validity, construct validity and criterion validity;
responsiveness is both a domain and a measurement property. Some measurement properties additionally contain multiple aspects;
for example, construct validity includes structural validity, hypothesis testing, and cross-cultural validity.

\newpage\subsection{Overview of the writing process}

\textbf{Prewriting} (Time proportion ~70\%)
\par\ \textbullet\ Collect, synthesize, and organize information;
\par\ \textbullet\ rainstorm take-home messages;
\par\ \textbullet\ Work out ideas away from the computer;
\par\ \textbullet\ Develop a road map/outline.

\textbf{Writing the first draft} (Time proportion ~10\%)
\par\ \textbullet\ Put your facts and ideas together in organized prose.

\textbf{Revision} (Time proportion ~20\%)
\par\ \textbullet\ Read your work out loud;
\par\ \textbullet\ Get rid of clutter;
\par\ \textbullet\ Do a verb check;
\par\ \textbullet\ Get feedback from others.

\newpage\subsection{The pre-writing step}
\textbf{Get organized first}
\par\ \textbullet\ Do not try to write and gather information simultaneously;
\par\ \textbullet\ Gather and organize information before writing the first draft!
\par\ \textbullet\ Extract key information when you read literature;
\par\ \textbullet\ Organizing your thoughts.

\textbf{Develop a road map}
\par\ \textbullet\ Arrange key facts and citations from the literature into a crude road map / outline before writing the first draft;
\par\ \textbullet\ Think in paragraphs and sections.

\textbf{Brainstorm away from the computer}
\par\ \textbullet\ Write on the go!
\par\ \textbullet\ Work out take-home messages;
\par\ \textbullet\ Organize your paper;
\par\ \textbullet\ Write memorable lines.
  
\textbf{Compositional organization}
\par\ \textbullet\ Like ideas should be grouped;
\par\ \textbullet\ Like paragraphs should be grouped;
\par\ \textbullet\ Do not ``Bait-and-Switch'' your reader too many times. When discussing a controversy, follow the:
\par\quad\textopenbullet\ Arguments (all);
\par\quad\textopenbullet\ Counter-arguments (all);
\par\quad\textopenbullet\ Rebuttals (all).

\newpage\subsection{The writing step}

\textbf{Do not be a perfectionist!}

\par\ \textbullet\ The goal of the first draft is to get the ideas down in complete sentences in order;

\par\ \textbullet\ Focus on logical organization more than sentence-level details;

\par\ \textbullet\ Writing the first draft is the hardest step for most people. Minimize the pain by writing the first draft quickly and efficiently.

\subsection{Revision}

\textbf{Read your work out loud}

\textbf{Do a verb check, underline the main verb in each sentence. Watch for:}
\par\ \textbullet\ Lackluster verbs;
\par\quad\textopenbullet\ There [are] many students who struggle with chemistry.
\par\ \textbullet\ Passive verbs;
\par\quad\textopenbullet\ The reaction [was] observed by her.
\par\ \textbullet\ Buried verbs: place the subject and verb early and close together.

\textbf{Cut the clutter: do not be afraid to cut! Watch for:}
\par\ \textbullet\ Deadweight words and phrases;
\par\quad\textopenbullet\ It should be emphasized that...
\par\ \textbullet\ Empty words and phrases;
\par\quad\textopenbullet\ basic tenets of, important...
\par\ \textbullet\ Long words or phrases that could be short;
\par\quad\textopenbullet\ muscular and cardiorespiratory performance...
\par\ \textbullet\ Unnecessary jargon and acronyms;
\par\ \textbullet\ Repetitive words or phrases;
\par\quad\textopenbullet\ teaches clinicians/guides clinicians
\par\ \textbullet\ Adverbs;
\par\quad\textopenbullet\ very, really, quite, basically...

\newpage\textbf{Do an organizational review;}
\par\ \textbullet\ In the margins of your paper, tag each paragraph with a phrase or sentence that sums up the main point;
\par\ \textbullet\ Then move paragraphs around to improve logical flow and bring similar ideas together.

\textbf{Get feedback from others;}
\par\ \textbullet\ Ask someone outside your department to read your manuscript;
\par\ \textbullet\ Without any technical background, they should easily grasp:
\par\quad\textopenbullet\ the main findings;
\par\quad\textopenbullet\ take-home messages;
\par\quad\textopenbullet\ significance of your work.
\par\ \textbullet\ Ask them to point out, particularly hard-to-read sentences and paragraphs.

\textbf{Get editing help.}
\par\ \textbullet\ Find a good editor to edit your work.

\newpage\quad

\newpage\section{Checklist for the final draft}

\textbf{One more tip on making writing easier}
\par\ \textbullet\ Break your writing task into small and realistic goals.
\par\quad\textopenbullet\ My goal is to write 400 words today.
\par\quad\textopenbullet\ My goal is to write the first two paragraphs of the discussion section today.

\textbf{Recommended order for writing and original manuscript}
\par\ 1. Tables and Figures
\par\ 2. Result
\par\ 3. Methods
\par\ 4. Introduction
\par\ 5. Discussion
\par\ 6. Abstract

\subsection{Tables and Figures}

\textbf{Tales and Figures are the foundation of your story.}
\par\ \textbullet\ Editors, reviewers, and readers may look first (and maybe only) at titles, abstracts, tables, and figures.
\par\ \textbullet\ Figures and tables should stand alone and tell a complete story. The reader should not need to refer back to the main text.

\textbf{Tips on Tables and Figures}
\par\ \textbullet\ Use the fewest figures and tables needed to tell the story;
\par\ \textbullet\ Do not present the same data in both a figure and a table.

\newpage\textbf{Tables features}
\par\ \textbullet\ Give precise values
\par\ \textbullet\ Display many values/variables

\textbf{Figures features}
\par\ \textbullet\ Visual impact;
\par\ \textbullet\ Show trends and patterns;
\par\ \textbullet\ Tell a quick story;
\par\ \textbullet\ Tell the whole story;
\par\ \textbullet\ Highlight a particular result.

\textbf{Table Title}
\par\ \textbullet\ Identify the specific topic or point of the table;
\par\ \textbullet\ Use the same key terms in the table title, the column headings, and the text of the paper.
\par\ \textbullet\ Keep it brief!
\par\quad\textopenbullet\ Descriptive characteristics of the two treatment groups, means ± SD or N(%)

\textbf{Table Footnotes}
\par\ \textbullet\ Use superscript symbols to identify footnotes, according to journal guidelines;
\par\quad\textopenbullet\ *, **, †, ‡, ††, ¶, \#, etc.
\par\ \textbullet\ Use footnotes to explain statistically significant differences;
\par\quad\textopenbullet\ * \textless p.01 vs. control by ANOVA
\par\ \textbullet\ Use footnotes to explain experiment details or abbreviations.
\par\quad\textopenbullet\ EDI is the Eating Disorder Inventory (reference)
\par\quad\textopenbullet\ Amenorrhea was defined as 0-3 periods per day.

\newpage\textbf{Table Formats}
\par\ \textbullet\ Model your tables from already published tables; Do not re-invent the wheel.
\par\ \textbullet\ Follow journal guidelines RE:
\par\quad\textopenbullet\ Roman or Arabic numbers
\par\quad\textopenbullet\ Centered or flush left table number, title, column, headings, and data
\par\quad\textopenbullet\ Capital letters and italics
\par\quad\textopenbullet\ The placement of footnotes
\par\quad\textopenbullet\ The type of footnote symbols
\par\ \textbullet\ Most journals use three horizontal lines:
\par\quad\textopenbullet\ One above the column heading;
\par\quad\textopenbullet\ One below the column heading;
\par\quad\textopenbullet\ One below the data.
\par\ \textbullet\ It is OK to use the shadow;
\par\ \textbullet\ Make sure everything lines up;
\par\ \textbullet\ Use a reasonable number of significant figures;
\par\ \textbullet\ Give units;
\par\ \textbullet\ Omit unnecessary columns;
\par\ \textbullet\ Unify the style!

\newpage\textbf{Classes of figures:}
\par\ \textbullet\ Image: A broad term for any visual representation.
\par\quad\textopenbullet\ Picture: A photo or drawing depicting real-world items or scenes.
\par\quad\textopenbullet\ Snapshot: An image captured at a moment in time; a photo or screen capture.
\par\quad\textopenbullet\ Rendering: A computer-generated image, often a 2D representation of 3D models.
\par\quad\textopenbullet\ Illustration: An image or artwork created to explain, decorate, or visually convey information.
\par\ \textbullet\ Diagram: A simplified graphic representing complex information.
\par\quad\textopenbullet\ Schematic: A simplified diagram showing components and relationships, often in systems or structures.
\par\quad\textopenbullet\ Blueprint: A detailed technical drawing, typically of architectural or engineering designs.
\par\ \textbullet\ Chart: A graphic representation of data.
\par\quad\textopenbullet\ **Graph: A diagram representing data, usually with lines, bars, or points.
\par\quad\textopenbullet\ Plot: A specialized chart displaying data points and relationships, e.g., scatter plots, and line plots.
\par\quad\textopenbullet\ Heatmap: A data visualization showing patterns through color variations, often for correlations or spatial distribution.
\par\quad\textopenbullet\ Infographic: A visual representation of information, using charts, images, and minimal text.
\par\ \textbullet\ Map: An image representing geographical areas, with terrain, landmarks, and borders.
\par\ \textbullet\ Figure: Any visual representation in scientific papers; including graphs, plots, images, or pictures. Used as a general term.

\textbf{Type of figures}
\par\ \textbullet\ Primary evidence
\par\quad\textopenbullet\ Experiment pictures
\par\quad\textopenbullet\ Indicates data quality
\par\quad\textopenbullet\ ``Seeing is believing''
\par\ \textbullet\ Graphs
\par\quad\textopenbullet\ line graphs, bar graphs, scatter plots, histograms, boxplots, etc.
\par\ \textbullet\ Drawings and diagrams
\par\quad\textopenbullet\ Illustrate an experimental set-up or work-flow;
\par\quad\textopenbullet\ Indicate flow of participants;
\par\quad\textopenbullet\ Illustrate cause-and-effect relationships or cycles;
\par\quad\textopenbullet\ Give a hypothetical model;
\par\quad\textopenbullet\ Represent microscopic particles or microorganisms as cartoons.

\textbf{Figure Legends: Allows the figure to stand alone, which may contain:}
\par\ 1. Brief title
\par\ 2. Essential experimental details
\par\ 3. Definitions of symbols or line/bar patterns
\par\ 4. Explanation of panels (Variables definitions, ...)
\par\ 5. Statistical information (test used, p-vales, ...)

\newpage\textbf{Graphs}
\par\ \textbullet\ Line graphs
\par\quad\textopenbullet\ Used to show trends over time, age, does, or other variables.
\par\ \textbullet\ Scatter graphs
\par\quad\textopenbullet\ Used to show relationships between two variables (particularly linear correlation);
\par\quad\textopenbullet\ Allows the reader to see individual data points: more information;
\par\quad\textopenbullet\ The lines on scatter graphs can draw your eye!
\par\ \textbullet\ Bar graphs
\par\quad\textopenbullet\ Used to compare groups at a time point;
\par\quad\textopenbullet\ Tells a quick visual story.
\par\ \textbullet\ Individual-value bar graphs
\par\ \textbullet\ Histograms
\par\ \textbullet\ Box plots
\par\ \textbullet\ Survival curves

\textbf{Tips for Graphs}
\par\ \textbullet\ Tell a quick visual story;
\par\ \textbullet\ Keep it simple;
\par\ \textbullet\ Make it easy to distinguish groups, e.g. triangles vs. circles vs. squares ...;
\par\ \textbullet\ If it is too complex, maybe it belongs in a table.

\newpage\textbf{Diagrams and Drawings}
\par\ \textbullet\ Illustrate an experimental set-up or work-flow;
\par\ \textbullet\ Indicate the flow of participants;
\par\ \textbullet\ Illustrate cause and effect relationships or cycles;
\par\ \textbullet\ Give a hypothetical model;
\par\ \textbullet\ Represent microscopic particles or microorganisms as cartoons.

\textbf{Besides tables and figures: Movies}
\par\ \textbullet\ Allowed as supplemental material.

\newpage\subsection{Results}

\textbf{Results != Raw data}
\par\ \textbullet\ The results section should:
\par\quad\textopenbullet\ Point out simple relationships;
\par\quad\textopenbullet\ Describe big-picture trends;
\par\quad\textopenbullet\ Cite figures or tables that present supporting data.
\par\ \textbullet\ Avoid simply repeating the numbers that are already available in tables and figures.

\par\ \textbullet\ Examples 1:
\par Over the course of treatment, topiramate was significantly more effective than placebo at improving drinking outcomes on drinks per day,
drinks per drinking day, percentage of heavy drinking days, percentage of days abstinent, and log plasma-glutamyl transferase ratio (Table 3).

\par\ \textbullet\ Examples 2:
\par The total suicide rates for Australian men and women did not change between 1991 and 2000 because marked decreases in older men and women (Table 1) were offset by increases in younger adults,
especially younger men.

\textbf{Tips for writing results}
\par\ \textbullet\ Break into subsections, with headings (if needed);
\par\ \textbullet\ Complement the information that is already in tables and figures:
\par\quad\textopenbullet\ Give precise values that are not available in the figure;
\par\quad\textopenbullet\ Report the percent change or percent difference if absolute values are given in the table;
\par\ \textbullet\ Repeat/highlight only the most important numbers;
\par\ \textbullet\ Do not forget to talk about negative and control results;
\par\ \textbullet\ Reserve the term ``significant'' for statistically significant;
\par\ \textbullet\ Reserve information about what you did for the methods section:
\par\quad\textopenbullet\ In particular, do not discuss the rationale for statistical analyses within the Result section;
\par\quad\textopenbullet\ The results section is about what you found, not what you did;
\par\ \textbullet\ Reserve comments on the meaning of your result for the discussion section.

The goal of the results section is to summarize trends in the data.

\textbf{What verb tense do I use?}
\par\ \textbullet\ Use past tense for complete actions:
\par\quad\textopenbullet\ We found that...
\par\quad\textopenbullet\ The average reaction time was...
\par\ \textbullet\ Use the present tense for assertions that continue to be true, such as what the tables show, what you believe, and what the data suggest
\par\quad\textopenbullet\ Figure 1 shows...
\par\quad\textopenbullet\ The finding confirms...
\par\quad\textopenbullet\ The data suggest...
\par\quad\textopenbullet\ We believe that...

\textbf{Use the active voice}
\par\ \textbullet\ More lively!
\par\ \textbullet\ Since you can talk about the subjects of your experiments, ``we'' can be used sparingly while maintaining the active voice.

\newpage\subsection{Practice writing results}

\par\ \textbullet\ Before:
\par The majority of runners ran during pregnancy (70.0\%, 77/110),
with 62.7\% running during the first trimester,
51.8\% during the second trimester, and fewer than one third (30.9\%) during the third trimester (Table 2).
From the 77 women who ran during pregnancy, we observed the average weekly mileage during pregnancy for those who ran to be 20.3 ± 9.3 miles.
Average running intensity was reported to be 47.9\% ± 21.0\% as a percent of non-pregnant running effort.
A small number (3.9\%, 3/77) reported sustaining a running injury while pregnant. About a quarter (24.8\%) waited 5-7 weeks to resume running post-partum.
A small fraction (5.7\%) resumed running less than a week after giving birth. Some women (11.4\%) waited more than six months post-partum to resume running.

\par\ \textbullet\ After:
\par Seventy percent of runners can during pregnancy (n=77), and almost one third ran during their third trimester (Table 2).
On average, those who ran greatly curtailed their training - running 20.3 ± 9.3 miles per week and cutting their intensity to half of their non-pregnant running effort.
Three reported sustaining a running injury while pregnant. In the post-partum period, nearly one quarter resumed running by two weeks after giving birth;
most resumed running by two months.

\newpage\subsection{Methods}

\textbf{Methods and Materials}
\par\ \textbullet\ Give a clear overview of what was done;
\par\ \textbullet\ Give enough information to replicate the study (like a recipe);
\par\ \textbullet\ Be complete, but make life easy for your reader:
\par\quad\textopenbullet\ Break into smaller sections with subheadings;
\par\quad\textopenbullet\ Cite a reference for commonly used methods;
\par\quad\textopenbullet\ Display in a flow diagram or table where possible;
\par\ \textbullet\ You may use jargon and the passive voice more liberally in the methods section.

\newpage\textbf{Who, what, when, where, how, and why questions to consider when writing the Methods section}
\par\ \textbullet\ Who
\par\quad\textopenbullet\ Who maintained the records?
\par\quad\textopenbullet\ Who reviewed the data?
\par\quad\textopenbullet\ Who collected the specimens?
\par\quad\textopenbullet\ Who enrolled the study participants?
\par\quad\textopenbullet\ Who supplied the reagents?
\par\quad\textopenbullet\ Who made the primary diagnosis?
\par\quad\textopenbullet\ Who did the statistical analysis?
\par\quad\textopenbullet\ Who reviewed the protocol for ethics approval?
\par\quad\textopenbullet\ Who provided the funding?
\par\ \textbullet\ What
\par\quad\textopenbullet\ What reagents, methods, and instruments were used?
\par\quad\textopenbullet\ What type of study was it?
\par\quad\textopenbullet\ What were the inclusion and exclusion criteria for enrolling study participants?
\par\quad\textopenbullet\ What protocol was followed?
\par\quad\textopenbullet\ What treatments were given?
\par\quad\textopenbullet\ What endpoints were measured?
\par\quad\textopenbullet\ What data transformation was performed?
\par\quad\textopenbullet\ What statistical software package was used?
\par\quad\textopenbullet\ What was the cutoff for statistical significance?
\par\quad\textopenbullet\ What control studies were performed?
\par\quad\textopenbullet\ What validation experiments were performed?
\newpage\par\ \textbullet\ When
\par\quad\textopenbullet\ When were specimens collected?
\par\quad\textopenbullet\ When were the analyses performed?
\par\quad\textopenbullet\ When was the study initiated?
\par\quad\textopenbullet\ When was the study terminated?
\par\quad\textopenbullet\ When were the diagnoses made?
\par\ \textbullet\ Where
\par\quad\textopenbullet\ Where were the records kept?
\par\quad\textopenbullet\ Where were the specimens analyzed?
\par\quad\textopenbullet\ Where were the study participants enrolled?
\par\quad\textopenbullet\ Where was the study performed?
\par\ \textbullet\ How
\par\quad\textopenbullet\ How were samples collected, processed, and stored?
\par\quad\textopenbullet\ How many replicates were performed?
\par\quad\textopenbullet\ How was the data reported?
\par\quad\textopenbullet\ How were the study participants selected?
\par\quad\textopenbullet\ How were patients recruited?
\par\quad\textopenbullet\ How was the sample size determined?
\par\quad\textopenbullet\ How were study participants assigned to groups?
\par\quad\textopenbullet\ How was response measured?
\par\quad\textopenbullet\ How were endpoints measured?
\par\quad\textopenbullet\ How were control and disease groups defined?
\par\ \textbullet\ Why
\par\quad\textopenbullet\ Why was a species chosen?
\par\quad\textopenbullet\ Why was a selected analytical method chosen?
\par\quad\textopenbullet\ Why was a selected experiment performed?
\par\quad\textopenbullet\ Why were experiments done in a certain order?

\newpage\textbf{Materials and Methods}
\par\ \textbullet\ Materials
\par\ \textbullet\ Participants/Subjects
\par\ \textbullet\ Experimental protocol/Study design
\par\ \textbullet\ Measurements
\par\quad\textopenbullet\ How were the dependent and independent variables measured?
\par\ \textbullet\ Analyses

\textbf{Make life easy for your reader.}
\par\ \textbullet\ Break into smaller sections with subheadings;
\par\ \textbullet\ Cite a reference for commonly used methods or previously used methods rather than explaining all the details;
\par\ \textbullet\ Display in a flow diagram or table where possible;

\textbf{Example subheadings}
\par\ \textbullet\ Methods
\par\quad\textopenbullet\ Subjects and experimental protocols
\par\quad\textopenbullet\ Hardware
\par\quad\textopenbullet\ GPS data processing
\par\quad\textopenbullet\ Wind

\textbf{Ver tense in the Method section}
\par\ \textbullet\ Report methods in the past tense:
\par\quad\textopenbullet\ We measured...
\par\ \textbullet\ But use present tense to describe how data are presented in the paper:
\par\quad\textopenbullet\ Data are summarized as mean ± SD.

\newpage It is OK to use the passive voice or even to use a combination

\textbf{Passive voice:} Emphasizes the method or variable:
\par\ \textbullet\ Oral temperatures were measured.

\textbf{Active voice:} More lively, but sacrifices having the material, method, or variable as the subject of the sentence;
It requires creativity to avoid starting every sentence with ``We''.
\par\ \textbullet\ We measured oral temperatures.

\newpage\subsection{Introduction}

\textbf{About Introduction}
\par\ \textbullet\ Good news: The introduction is easier to write than you may realize;
\par\ \textbullet\ Follows a fairly standard format;
\par\ \textbullet\ Typically 3 paragraphs long;
\par\quad\textopenbullet\ Recommended range: 3 to 5;
\par\ \textbullet\ It is not an exhaustive review of your general topic:
\par\quad\textopenbullet\ Should focus on the specific hypothesis/aim of your study.

\textbf{The Order of Introduction}
\par\ \textbullet\ Background, know information
\par\ \textbullet\ Knowledge gap, unknown information
\par\ \textbullet\ Hypothesis, question, purpose statement
\par\ \textbullet\ Approach, plan of attack, and proposed solution

\textbf{Corresponds to roughly 3 - 5 paragraphs}
\par\ \textbullet\ Paragraph 1: What's known
\par\ \textbullet\ Paragraph 2: What's unknown
\par\quad\textopenbullet\ Limitations and gaps in previous studies
\par\ \textbullet\ Paragraph 3:
\par\quad\textopenbullet\ Your burning question, hypothesis, or aim
\par\quad\textopenbullet\ Your experimental approach
\par\ \textbullet\ Paragraph 4:
\par\quad\textopenbullet\ Why your experimental approach is new, different, or important (fills in the gaps)

\newpage\textbf{Tips for Writing an Introduction}
\par\ \textbullet\ Keep paragraphs short;
\par\ \textbullet\ Write for a general audience:
\par\quad\textopenbullet\ Clear, concise, non-technical
\par\ \textbullet\ Take the reader step by step from what is known to what is unknown. End with your specific question;
\par\quad\textopenbullet\ Known → Unknown → Question or Hypothesis
\par\ \textbullet\ Emphasize how your study fills in the gaps (the unknown);
\par\ \textbullet\ Explicitly state your research question, aim, or hypothesis;
\par\quad\textopenbullet\ We asked whether...
\par\quad\textopenbullet\ Our hypothesis was...
\par\quad\textopenbullet\ We tested the hypothesis that...
\par\quad\textopenbullet\ Our aims were...
\par\ \textbullet\ Do not answer the research question (no result or implications)!
\par\ \textbullet\ Summarize at a high level. Leave detailed descriptions, speculations, and criticisms of particular studies for the discussion.

\newpage\subsection{Discussion}

\textbf{Features of Discussion}
\par\ \textbullet\ Gives you the most freedom;
\par\ \textbullet\ Gives you the most chance to put good writing on display;
\par\ \textbullet\ Is the most challenging to write.

\textbf{The Order of Discussion}
\par\ \textbullet\ Answer the question asked;
\par\ \textbullet\ Support your conclusion (your data, other's data);
\par\ \textbullet\ Defend your conclusion (anticipate criticisms);
\par\ \textbullet\ Give the ``big-picture'' take-home message.

\newpage\textbf{Organize the Discussion}
\par\ \textbullet\ Key finding (answer to the questions asked in the Introduction):
\par\quad\textopenbullet\ Start with: ``We found that...''(or something similar);
\par\quad\textopenbullet\ Explain what data mean (big-picture);
\par\quad\textopenbullet\ State if the findings are novel.
\par\ \textbullet\ Key secondary findings
\par\ \textbullet\ Context
\par\quad\textopenbullet\ Give possible mechanisms or pathways;
\par\quad\textopenbullet\ Compare your results with other's results;
\par\quad\textopenbullet\ Discuss how your findings support or challenge the paradigm.
\par\ \textbullet\ Strengths and limitations:
\par\quad\textopenbullet\ Anticipate readers' questions or criticisms;
\par\quad\textopenbullet\ Explain why your results are robust;
\par\ \textbullet\ What's next:
\par\quad\textopenbullet\ Recommended confirmatory studies (what is needed to be confirmed.);
\par\quad\textopenbullet\ Point out unanswered questions and future directions.
\par\ \textbullet\ The ``So what?'': implicate, speculate, recommend:
\par\quad\textopenbullet\ Give the big-picture (human) implications of basic science findings;
\par\quad\textopenbullet\ Tell readers why they should care.
\par\ \textbullet\ Strong conclusion
\par\quad\textopenbullet\ Restate your main findings;
\par\quad\textopenbullet\ Give a final take-home message.

\newpage\textbf{Tips for Writing a Discussion}
\par\ \textbullet\ Showcase good writing!
\par\quad\textopenbullet\ Use the active voice;
\par\quad\textopenbullet\ Tell it like a story.
\par\ \textbullet\ Start and end with the main finding;
\par\quad\textopenbullet\ We found that...
\par\ \textbullet\ Do not travel too far from your data;
\par\quad\textopenbullet\ Focus on what your data do prove, not what you had hoped your data would prove!
\par\ \textbullet\ Focus on the limitations that matter, not generic limitations (be specific);
\par\ \textbullet\ Make sure your take-home message is clear and consistent.

\textbf{What NOT to do}
\par Do not start your discussion section with the limitations;

\textbf{Verb tense in the discussion section.}
\par\ \textbullet\ The past voice: when referring to study details, results, analyses, and background research:
\par\quad\textopenbullet\ We found that...
\par\quad\textopenbullet\ Subjects may have experienced...
\par\quad\textopenbullet\ Miller et al. found...
\par\ \textbullet\ The present voice: when talking about what the data suggest:

\newpage\subsection{Abstract}

\textbf{About Abstract}
\par\ \textbullet\ Abstract means ``to pull out'':
\par\quad\textopenbullet\ Overview of the main story;
\par\quad\textopenbullet\ Gives highlights from each section of the paper;
\par\quad\textopenbullet\ Limited length: 100 - 300 words, typically;
\par\quad\textopenbullet\ Stands on its own;
\par\quad\textopenbullet\ Most often, the only part people read;
\par\quad\textopenbullet\ Write the abstract after writing the paper.

\textbf{Abstract components}
\par\ \textbullet\ Background;
\par\ \textbullet\ Question/aim/hypothesis
\par\quad\textopenbullet\ We asked whether, ...
\par\quad\textopenbullet\ We hypothesized that...
\par\ \textbullet\ Experiments
\par\quad\textopenbullet\ Quick summary of key materials and methods.
\par\ \textbullet\ Results
\par\quad\textopenbullet\ Key results found;
\par\quad\textopenbullet\ Minimal raw data (prefer summaries).
\par\ \textbullet\ Conclusion
\par\quad\textopenbullet\ The answer to the question asked, or take-home message.
\par\ \textbullet\ Ends of abstract
\par\quad\textopenbullet\ Implication, speculation, or recommendation.

\textbf{The two forms of Abstract}
\par The abstract may be structured (with subheadings), or free form.

\newpage\quad

\newpage\section{Publication process}

\subsection{Plagiarism}

\textbf{Plagiarism of others' work}
\par\ \textbullet\ Passing off others' writing (or table and figures) as your own:
\par\quad\textopenbullet\ Cutting and pasting sentences or even phrases from another source;
\par\quad\textopenbullet\ Slightly rewriting or re-arranging others' words;
\par\quad\textopenbullet\ ``Borrowing'' material from sites like Wikipedia without citing.

\textbf{When writing about others' ideas or work:}
\par\ \textbullet\ You need to understand the material well enough to put it in your own words;
\par\ \textbullet\ Work from memory;
\par\ \textbullet\ Draw your conclusion;
\par\ \textbullet\ Do NOT mimic the original authors' sentence structure or just re-arrange the original author's words.

\textbf{Self-plagiarism and duplication}
\par\ \textbullet\ Recycling your previous writing or data, including:
\par\quad\textopenbullet\ Copying or only slightly rewriting text from your previously published papers;
\par\quad\textopenbullet\ Adding new data to already published data and presenting it as new results;
\par\quad\textopenbullet\ Submitting identical or overlapping data to multiple journals.

\newpage\subsection{Authorship}

\textbf{Who gets authorship?}
\par\ \textbullet\ Any author listed on the paper's title page should take public responsibility for its content;

\textbf{In what order?}
\par\ \textbullet\ Order implies authors' relative contributions (except for the senior author position).;
\par\ \textbullet\ The senior author (head of the lab or research team) often appears as the last-listed author;
\par\ \textbullet\ Papers may have dual first authors;
\par\ \textbullet\ For fairness, alphabetical or reverse alphabetical order may be used if researchers have contributed equally;
\par\ \textbullet\ Large working groups may be cited as a group.

\textbf{Acknowledgments}
\par\ \textbullet\  Funding sources
\par\ \textbullet\ Contributors who did not get authorship (e.g. offered materials, advice, or consultation that were not significant enough to merit authorship).

\textbf{Ghost authors}
\par\ \textbullet\ Writers-for-hire who draft manuscripts (usually for companies), but are not listed as authors.

\textbf{Guest or ``honorary'' authors}
\par\ \textbullet\ Academic researchers who are minimally involved in a paper, but ``lend'' their name as an author (often first author) to bolster the paper's credibility.

\newpage\subsection{The Submission Process}

\textbf{Submission process}
\par\ \textbullet\ Identify a journal for submission (ideally before writing);
\par\ \textbullet\ Follow the online ``instruction for authors'' for writing and formatting the manuscript;
\par\ \textbullet\ Submit your manuscript online (corresponding author):
\par\quad\textopenbullet\ All authors must fill out and sign copyright transfer and conflict, or interest forms (often done offline);
\par\ \textbullet\ Possible outcomes:
\par\quad\textopenbullet\ accepted
\par\quad\textopenbullet\ accepted pending minor revisions
\par\quad\textopenbullet\ rejected but re-submission is possible
\par\quad\textopenbullet\ rejected and no resubmission is possible
\par\ \textbullet\ Revision and resubmission:
\par\quad\textopenbullet\ re-submit with a cover letter that addresses the reviewer's critiques point by point.
\par\ \textbullet\ Once accepted, carefully review the final proofs.

\textbf{When responding to reviewers' critiques of your paper:}
\par\ \textbullet\ You should respond to every criticism, even if you don't make the change that the reviewer requested.

\subsection{Interview with Dr. Bradley Efron}

\par Writing does not necessarily require ornate writing skills.

\newpage\subsection{Interview with Dr. George Lundberg}

\textbf{Avoid list}
\par\ \textbullet\ Submitting an article to journals poorly suited to the piece;
\par\ \textbullet\ Writing an article that is too long;
\par\ \textbullet\ Ignoring the instructions to authors;
\par\ \textbullet\ Drawing conclusions that go beyond the data.

\textbf{Increase the authors' chances of getting published}
\par\ \textbullet\ Be appropriately humble;
\par\ \textbullet\ Be willing to take chances;
\par\ \textbullet\ Be ambitious, yet also realistic;
\par\ \textbullet\ Find the right journal for your work;
\par\ \textbullet\ Follow the instructions for authors;
\par\ \textbullet\ Allow time for editing and collaboration.

\textbf{Advice to young scientists}
\par\ \textbullet\ Respect the process, but maintain your confidence;

\textbf{About rejection}
\par\ \textbullet\ Rejection happens to everyone;
\par\ \textbullet\ High-impact journals reject unsolicited manuscripts at a rate of 95%.

\textbf{Tips for resubmission}
\par\ \textbullet\ Be grateful that the journal is interested;
\par\ \textbullet\ Carefully consider the reviewers' comments;
\par\ \textbullet\ Either make the changes or argue effectively as to why you did not;
\par\ \textbullet\ Respond via a cover letter for the revised manuscript.

\textbf{Anticipate occurring in the publication process over the next decade}
\par\ \textbullet\ Open-access publishing in science and medicine will become the rule;

\textbf{The last tips}
\par\ \textbullet\ Follow the instructions;
\par\ \textbullet\ Choose a journal that fits your study;
\par\ \textbullet\ Do what the editor wants you to do;
\par\ \textbullet\ Rely on collaborators that you trust.

\subsection{Interview with Dr. Gary Friedman}

\textbf{Key elements for journal editors looking for}
\par\ \textbullet\ Novelty - Avoid repeating findings that are already well-known;
\par\ \textbullet\ Interest to readers;
\par\ \textbullet\ Good writing or good American English;
\par\ \textbullet\ Concise writing.

\textbf{The mistake that scientists make when submitting their paper}
\par\ \textbullet\ Over-valuation of one's paper.

\textbf{Some advice}
\par\ \textbullet\ Be clear and concise;
\par\ \textbullet\ Avoid repetition;
\par\ \textbullet\ Avoid wordiness;
\par\ \textbullet\ Keep the introduction brief;
\par\ \textbullet\ Do not repeat numerical information from the Tables in the Results section;
\par\ \textbullet\ You won't dumb down the science;
\par\ \textbullet\ Papers should be understandable to an educated audience.

\textbf{Advice for the first-time authors}
\par\ \textbullet\ A paper should not read like a thesis;
\par\ \textbullet\ Write like a senior scientist;
\par\ \textbullet\ State only the important strengths and limitations relevant to the study.

\newpage\textbf{About resubmission}
\par\ \textbullet\ Be encouraged by the opportunity to resubmit;
\par\ \textbullet\ List and respond to every comment separately;
\par\ \textbullet\ Fix the problem or explain why you did not fix it;
\par\ \textbullet\ Show the editor where you made the changes;
\par\ \textbullet\ Be polite.

\textbf{About rejection}
\par\ \textbullet\ Fix it and submit it elsewhere;
\par\ \textbullet\ Do not take rejection personally;
\par\ \textbullet\ Understand that reviewers may be from different disciplines;
\par\ \textbullet\ Remember that reviewers can have bad days too.

\textbf{Key changes occur in the publication process over the next decade}
\par\ \textbullet\ More online publications;
\par\ \textbullet\ Faster publication;
\par\ \textbullet\ More new journals of uncertain quality.

\newpage\subsection{Doing a peer view}

If you are the reviewer, a few tips as follows

\textbf{Tone}
\par\ \textbullet\ Assume there is some poor graduate student on the other end who did all the work, and whose confidence and career depend on your critique;
\par\ \textbullet\ Tone Matters!
\par\ \textbullet\ Avoid criticizing the authors, criticize the work;
\par\ \textbullet\ Avoid generalizations; point out specific errors;
\par\ \textbullet\ Avoid ``lecturing'' to the authors;
\par\ \textbullet\ Use positive instead of negative language where possible;

\textbf{Types of peer review}
\par\ \textbullet\ Single-blind: Most common; authors are blinded to the reviewers;
\par\ \textbullet\ Double-blind: Reviewers are additionally blinded to authors;
\par\ \textbullet\ Open: Neither reviewers nor authors are blinded; reviewers' names (and reviews) may be publicly available;
\par\ \textbullet\ Post-publication peer review: Blogs, comments, etc.

\textbf{Process of peer review}

\par\ \textbullet\ Scan the abstract;

\par\ \textbullet\ Jump to the data: review the tables and figures first:
\par\quad\textopenbullet\ Draw your conclusions;
\par\quad\textopenbullet\ Do the tables and figures stand on their own?
\par\quad\textopenbullet\ Are there any obvious statistical errors?
\par\quad\textopenbullet\ Is there repetitive information?

\newpage\par\ \textbullet\ Read the paper once through:
\par\quad\textopenbullet\ Do the author's conclusions match their data?
\par\quad\textopenbullet\ Is the paper written clearly, or did you struggle to get through it? You should not have to struggle;
\par\quad\textopenbullet\ Is the length of the paper justified given the amount of new information that the data provide?

\par\ \textbullet\ Read the introduction carefully:
\par\quad\textopenbullet\ Is it sufficiently succinct?
\par\quad\textopenbullet\ Does it roughly follow: know → unknown → research question/hypothesis?
\par\quad\textopenbullet\ Is there a clear statement of the hypotheses or aim of the study?
\par\quad\textopenbullet\ Is there detailed information about what was done that belongs in the methods?
\par\quad\textopenbullet\ Is there information about what was found that belongs in the results?
\par\quad\textopenbullet\ Is there distracting information about previous studies or mechanisms that are not directly relevant to the hypothesis being tested?
If so, it should be moved to discussion.
\par\quad\textopenbullet\ Do the authors tell you what gaps in the literature they are trying to fill in?

\par\ \textbullet\ Read the methods carefully:
\par\quad\textopenbullet\ Scan the section to find answers to your questions about the data;
\par\quad\textopenbullet\ Were things measured objectively or subjectively? What instruments were used?
\par\quad\textopenbullet\ Are there flaws in the study design? Such as no control group?
\par\quad\textopenbullet\ Read the statistics section carefully;

\newpage\par\ \textbullet\ Read the results carefully:
\par\quad\textopenbullet\ Read this section with the tables and figures in front of you;
\par\quad\textopenbullet\ Does each section roughly correspond to one table or figure?
\par\quad\textopenbullet\ Do the authors summarize the main trends and themes from the table, or do they just repeat what is in the table?
\par\quad\textopenbullet\ If there are graphs, do they try to draw your eye to what they want you to see?
\par\quad\textopenbullet\ Does the author over-interpret statistical significance,
by ignoring the fact that the magnitude is small or by ignoring the fact that they have done multiple subgroup analyses?
\par\quad\textopenbullet\ Is this section unnecessarily long?

\par\ \textbullet\ Look at each table and figure:
\par\quad\textopenbullet\ Did the authors choose the correct statistics?
\par\quad\textopenbullet\ Are there multiple tables or figures that tell the same story? For example,
Table 1 gives the mean values for two groups and indicates statistical significance from a test and Table 2 gives confidence intervals for the differences in means for the same data;
\par\quad\textopenbullet\ Is there evidence of cherry-picking or purposefully omitting data?
\par\quad\textopenbullet\ Are any graphs misleading, e.g. through manipulation of area or axes?
\par\quad\textopenbullet\ Is the ``treatment'' group always compared with a proper control or placebo group?
\par\quad\textopenbullet\ Are there inconsistencies in the data they present from one table to the next?
\par\quad\textopenbullet\ Did the authors make transcribing errors when going from the data in tables or results to the abstract?

\newpage\par\ \textbullet\ Read the discussion carefully:
\par\quad\textopenbullet\ Does the first paragraph tell you succinctly and clearly what was found and what is new?
\par\quad\textopenbullet\ Are the authors' conclusions justified or are they overreaching?
\par\quad\textopenbullet\ Do they clearly distinguish hypothesis-driven conclusions and exploratory conclusions?
\par\quad\textopenbullet\ Is the writing clear and to the point (active voice)? Is there some sense of order and structure or are they just rambling on aimlessly?
\par\quad\textopenbullet\ Could the discussion be shortened?
\par\quad\textopenbullet\ Did they address the limitations you care about? (as opposed to any old irrelevant limitations that they threw in just to have some.)
\par\quad\textopenbullet\ Are the references that they cite current?
\par\quad\textopenbullet\ Have they omitted key references?

\newpage\subsection{Content of peer review}

\textbf{Comments to authors:}
\par\ \textbullet\ Start with a one-paragraph ``general overview'':
\par\quad\textopenbullet\ State what you think is the major finding and importance of the work;
\par\quad\textopenbullet\ Give 2~3 positive, encouraging statements about the work.
\par\quad\textopenbullet\ State 1~2 major limitations;
\par\quad\textopenbullet\ Do not tell the authors your overall recommendation (rejection, acceptance).
\par\ \textbullet\ In a numbered list, give 5~15 specific criticisms or suggestions for revision.
The number will often correspond to your recommendation (give the most if you are recommending an ``opportunity for revision''):
\par\quad\textopenbullet\ Point out specific mistakes;
\par\quad\textopenbullet\ List the issues that you found in your review;
\par\quad\textopenbullet\ Give specific recommendations for revision.

\textbf{Reviewer != copy editor:}
\par\ \textbullet\ Do not spend your time nit-picking;
\par\ \textbullet\ Focus on big-picture issues;
\par\ \textbullet\ If the manuscript has a lot of copy-editing errors, point this out in a general way and give one or two examples.
\par\quad\textopenbullet\ The manuscript contains typos, such as...

\textbf{Comments to editors (authors do not see these):}
\par\ \textbullet\ Fill out the journal ``grading sheet'' (if applicable);
\par\ \textbullet\ Choose your recommendation:
\par\quad\textopenbullet\ Reject;
\par\quad\textopenbullet\ Reject with an opportunity to revise;
\par\quad\textopenbullet\ Accept with minor revisions;
\par\quad\textopenbullet\ Accept.
\par\ \textbullet\ Give a succinct overall statement to the editors that justifies your ranking. Be frank with the editors about your opinion and concerns.

\newpage\subsection{Predatory journals}

\textbf{Predatory journals}
\par\ \textbullet\ Predatory journals are bogus journals that are exploiting the open-access model to make money;
\par\ \textbullet\ They publish any garbage, without any peer review, simply to be able to collect the publishing fee from the authors.

\textbf{Checklist for avoiding predatory journals (Declan Butler)}
\par\ \textbullet\ Check that the publisher provides full, verifiable contact information, including address,
on the journal site. Be cautious of those that provide only web contact forms;
\par\ \textbullet\ Check that a journal's editorial board lists recognized experts with full affiliations.
Contact some of them and ask about their experience with the journal or publisher;
\par\ \textbullet\ Check that the journal prominently displays its policy for author fees;
\par\ \textbullet\ Be wary of e-mail invitations to submit to journals or to become editorial board members;
\par\ \textbullet\ Read some of the journal's published articles and assess their quality. Contact past authors to ask about their experience;
\par\ \textbullet\ Check that a journal's peer-review process is clearly described and try to confirm that a claimed impact factor is correct;
\par\ \textbullet\ Find out whether the journal is a member of an industry association that vets its members,
such as the Directory of Open Access Journals \textless www.doaj.org \textgreater, or the Open Acess Scholarly Publishers Association \textless www.oaspa.org\textgreater;
\par\ \textbullet\ Use common sense, as you would when shopping online: if something looks fishy, process with caution;
\par\ \textbullet\ Consider the H-Index or Impact Factor (IF) of the journal;
\par\ \textbullet\ Alert the false indicators, such as Universal Impact Factor (UIF), Global Impact Factor (GIF), or Journal Impact Factor (JIF).

\newpage\section{Review}

\subsection{Writing a review article}

\textbf{Goals of review articles}
\par\ \textbullet\ Synthesize and evaluate the recent primary literature on a topic;
\par\ \textbullet\ Summarize the current state of knowledge on a topic;
\par\ \textbullet\ Address controversies;
\par\ \textbullet\ Provide a comprehensive list of citations.

\textbf{Type I: Non-systematic review}
\par\ \textbullet\ Sometimes called a ``narrative'' review;
\par\ \textbullet\ May not be comprehensive;
\par\ \textbullet\ Qualitative review.

\textbf{Type II: Systematic review}
\par\ \textbullet\ Attempt to find and summarize all relevant studies. May even include unpublished work;
\par\ \textbullet\ Follow a rigorous search strategy using pre-defined exclusion and inclusion criteria. Searches multiple databases;
\par\ \textbullet\ Evaluate the quality of each study using rigorous, pre-defined criteria (often quantitative).

\textbf{Type III: Meta-analysis}
\par\ \textbullet\ A systematic review that additionally uses statistical techniques to pool data from independent studies (sometimes including unpublished studies).

\textbf{Tips for Review articles}
\par\ \textbullet\ Start with a more broad search, and then narrow it;
\par\ \textbullet\ Clearly define your thesis or theme;
\par\ \textbullet\ Invest time in getting organized;
\par\ \textbullet\ Divide the review into sections with separate headings;
\par\ \textbullet\ Consider putting information in tables, figures, and/or sidebar;
\par\ \textbullet\ Write for a broad audience.

\newpage\subsection{Grants I}

\textbf{Why submit a research proposal}
\par\ \textbullet\ Clarifies and deepens your thinking;
\par\ \textbullet\ Increases productivity and impact;
\par\ \textbullet\ Critical in all career paths;
\par\ \textbullet\ Securing funding is an accomplishment and has positive career benefits;

\textbf{Tip 1: Start early and gather critical information}
\par\ \textbullet\ Compile possible funding opportunities;
\par\ \textbullet\ Gather critical information:
\par\quad\textopenbullet\ Instructions;
\par\quad\textopenbullet\ Funded and unfunded examples.

\textbf{Tip 2: Create a game plan and write regularly}
\par\ \textbullet\ Create a task list;
\par\ \textbullet\ Write regularly;

\textbf{Tip 3: Find your research niche}
\par\ \textbullet\ Deep awareness of your field - identify critical knowledge gaps;
\par\ \textbullet\ Broad familiarity with the wider scientific community;

\textbf{Tip 4: Use your specific aims document as your roadmap}
\par\ \textbullet\ Specific aims:
\par\quad\textopenbullet\ Is the question important?
\par\quad\textopenbullet\ What is the overall goal?
\par\quad\textopenbullet\ What specifically will be done?
\par\quad\textopenbullet\ What is the expected payoff?

\textbf{Tip 5: Build a first-rate team}
\par\ \textbullet\ Leverage the strengths and expertise of collaborators (Team science);
\par\ \textbullet\ Cross-disciplinary to accelerate scientific innovation;

\newpage\textbf{Tip 6: Develop a complete research plan}
\par\ \textbullet\ Is there a need?
\par\ \textbullet\ How will the specific aims be accomplished?
\par\ \textbullet\ How long will the project take?
\par\ \textbullet\ What is next?

\textbf{Tip 7 STOP! Get feedback!}

\textbf{Tip 8 Tell a consistent and cohesive story}
\par\ \textbullet\ Grants have numerous documents or sections;
\par\ \textbullet\ Must tell a consistent and cohesive story.

\textbf{Tip 9: Follow specific requirements and proofread for errors and readability}
\par\ \textbullet\ Strictly follow specific formats and page requirements;
\par\ \textbullet\ Proofread!

\textbf{Tip 10: Recycle and resubmit}
\par\ \textbullet\ Recycle: submit to many funding opportunities;
\par\ \textbullet\ Resubmit: try again.

\subsection{Grants II}

\textbf{Why is the specific aims document important}
\par\ \textbullet\ Perfect for eliciting feedback;
\par\ \textbullet\ t is a roadmap;
\par\ \textbullet\ he reviewer will read it.

\textbf{Specific aims instructions}

\newpage\par\textbullet\ An example:
\par State concisely the goals of the proposed research and summarize the expected outcomes, including the impact that the results of the proposed research will exert on the research fields involved.
\par List succinctly the specific objectives of the research proposed, e.g., to test a stated hypothesis, create a novel design, solve a specific problem, challenge an existing paradigm or clinical practice, address critical barrier progress in the field, or develop new technology.

\textbf{Key points}
\par\ \textbullet\ Is the question important?
\par\quad\textopenbullet\ Attention-grabbing the first sentence;
\par\quad\textopenbullet\ Bring reviewers up to speed;
\par\quad\textopenbullet\ Frame the knowledge gap/need.
\par\ \textbullet\ What is the overall goal?
\par\quad\textopenbullet\ Big-picture goal,
\par\quad\textopenbullet\ The objective of this proposal,
\par\quad\textopenbullet\ Best bet or hypothesis,
\par\quad\textopenbullet\ Supportive preliminary data.
\par\ \textbullet\ What specifically will be done?
\par\quad\textopenbullet\ Aims,
\par\quad\textopenbullet\ Working hypotheses,
\par\quad\textopenbullet\ Methods.
\par\ \textbullet\ What is the expected payoff?
\par\quad\textopenbullet\ Return on investment,
\par\quad\textopenbullet\ Related to goals of the funding announcement.

\newpage\subsection{Grants III}

Use your specific aims pages as a road map for developing a strong research plan.

\textbf{Key questions for your research proposal}
\par\ \textbullet\ Is there a need?
\par\ \textbullet\ How will the project be accomplished?
\par\quad\textopenbullet\ What methods and analyses will be used;
\par\quad\textopenbullet\ What are the expected outcomes?
\par\quad\textopenbullet\ What might go wrong and how will it be managed?
\par\quad\textopenbullet\ What are the alternative approaches?
\par\ \textbullet\ How long will the project take?

\textbf{Note concerning innovation}
\par\ \textbullet\ Innovation:
\par\quad\textopenbullet\ Approaches,
\par\quad\textopenbullet\ Methodology,
\par\quad\textopenbullet\ Describe how your proposal improves upon previous research or technology.

\textbf{An outline for your research proposal}
\par\ \textbullet\ Background or significance,
\par\ \textbullet\ Aims,
\par\ \textbullet\ Timeline,
\par\ \textbullet\ Conclusion and future directions.

\textbf{Background or Significance}
\par\ \textbullet\ Importance of the problem,
\par\ \textbullet\ The premise for the proposed project, including strengths or weaknesses of published research pr preliminary data,
\par\ \textbullet\ Outline the knowledge gap or technical deficiency that the project will overcome.

\newpage\textbf{Aims}
\par\ \textbullet\ A hypothesis,
\par\ \textbullet\ Specific aims and objectives used to examine the hypothesis,
\par\ \textbullet\ The description of methods, approaches, or techniques to be used,
\par\ \textbullet\ The discussion of possible problems and how they will be managed,
\par\ \textbullet\ Alternative approaches might be tried.

\textbf{Organizing each aim}
\par\ \textbullet\ introduction,
\par\ \textbullet\ Preliminary data,
\par\ \textbullet\ Methods,
\par\ \textbullet\ Expect outcomes,
\par\ \textbullet\ Alternative approaches.

\textbf{Introduction}
\par\ \textbullet\ Provide an overview of the aim, including specific objectives, working hypothesis, rationale, and expected outcomes.

\textbf{Preliminary data}
\par\ \textbullet\ The critical review of the relevant literature;
\par\ \textbullet\ Preliminary studies:
\par\quad\textopenbullet\ Establish the feasibility of the project;
\par\quad\textopenbullet\ Clear and able to stand alone.
\par\ \textbullet\ Lead reviewer to conclude that you and the project are capable of success.

\textbf{Methods}
\par\ \textbullet\ Provide a detailed description of the experimental design including:
\par\quad\textopenbullet\ Validation of essential reagents or approaches,
\par\quad\textopenbullet\ Description of control and their significance,
\par\quad\textopenbullet\ Statistical analysis and interpretation.

\newpage\textbf{Expect outcomes}
\par\ \textbullet\ Summarize expected experimental outcomes and provide an interpretation of the data;
\par\ \textbullet\ What is the immediate payoff?
\par\ \textbullet\ Does this address the knowledge gap you wish to bridge?

\textbf{Alternative approaches}
\par\ \textbullet\ Introduction of alternative approaches by highlighting potential problems.

\textbf{Timeline}
\par\ \textbullet\ The timeline can demonstrate feasibility.

\textbf{Conclusion and future directions}
\par\ \textbullet\ Summarize expected outcomes, how they will bridge a current knowledge gap, and how the proposed project will lead to progress in the field,
\par\ \textbullet\ Discuss future experiments or approaches.

\newpage\subsection{Write letters of recommendation}

\textbf{Things to consider}
\par\ \textbullet\ It's OK to decline if you cannot write a strong letter;
\par\ \textbullet\ Take into account the competitiveness of the position or award;
\par\ \textbullet\ Never ask students to draft their recommendation letters.

\textbf{The candidate should provide}
\par\ \textbullet\ CV or resume,
\par\ \textbullet\ Information about the position or award,
\par\ \textbullet\ The deadline,
\par\ \textbullet\ Specific information about how to submit the letter.

\textbf{Formatting}
\par\ \textbullet\ Format it like an old-fashioned letter (date, address of the committee, etc.);
\par\ \textbullet\ Use letterhead or the electronic version;
\par\ \textbullet\ Avoid generic greetings such as ``to whom it may concern''. 

\par Rather, address it to a person (if known) or ``\textless instruction\_name\textgreater\ admissions committee '' or ``\textless instruction\_name\textgreater\ scholarship committee ''.

\textbf{First paragraph: Introduction}
\par\ \textbullet\ ``I am pleased/delighted/thrilled to recommend \textless student\_name\textgreater\ for \textless instruction\_name\textgreater.''
\par\ \textbullet\ or ``It's a pleasure to recommend \textless student\_name\textgreater\ for \textless instruction\_name\textgreater.''
\par\ \textbullet\ How do you know the candidate? How long have you known the candidate?
\par\ \textbullet\ 1-2 sentence overview:
\par\quad\textopenbullet\ Highest praise: ``She is one of the most brilliant and accomplished students that I have taught to date.''
\par\quad\textopenbullet\ Typical praise: ``I've found her to be a diligent student and researcher. I'm confident that she would be an asset to your research team.''

\newpage\textbf{Body of the letter}
\par\ \textbullet\ Use clear, concise, engaging language;
\par\ \textbullet\ The length of the letter matters;
\par\ \textbullet\ Address qualities relevant to the position/award, such as:
\par\quad\textopenbullet\ Quantitative skills,
\par\quad\textopenbullet\ Communication skills,
\par\quad\textopenbullet\ Ability to work with others,
\par\quad\textopenbullet\ Initiative,
\par\quad\textopenbullet\ Ability to prioritize tasks,
\par\quad\textopenbullet\ Creativity,
\par\quad\textopenbullet\ Attributes of a ``good citizen''.
\par\ \textbullet\ Give specific examples and stories, ``Show, don't tell'';
\par\ \textbullet\ Quantify and compare;
\par\quad\textopenbullet\ She is among the top than percent of MS students I have taught at Stanford.
\par\ \textbullet\ Point out extenuating circumstances (if applicable), and emphasize the highlights;
\par\ \textbullet\ If possible, quote others:
\par\quad\textopenbullet\ One of my students wrote this in an unsolicited email: ``You've probably heard already that \textless student\_name\textgreater\  has been a fantastic TA.''

\textbf{Be cautious}
\par\ \textbullet\ When a letter focuses more on the recommender, class, or project than on the candidate, this is a red flag.
\par\ \textbullet\ It's OK to highlight strengths from the student's CV, but do not simply repeat what is on the CV.

\newpage\textbf{Last paragraph: Concluding}
\par\ \textbullet\ In summary...
\par\ \textbullet\ Highest praise: ``In summary, \textless student\_name\textgreater\ is a star in all aspects. If there is anything else I can do to support her application, please do not hesitate to contact me.''
\par\ \textbullet\ Typical praise: ``I highly recommend \textless student\_name\textgreater\ for this position. If you have any further questions, I would be happy to expand further on my comments.''

\textbf{An example of hidden Language I}
\par\ \textbullet\ \textbf{A}: ``He is one of the best students I've had in my career at Stanford.''
\par\ \textbullet\ \textbf{B}: ``He was one of the best students in my class of 50.''
\par\ \textbullet\ \textbf{C}: ``He was the best student in my class of 50.''
\par\ \textbullet\ \textbf{D}: ``Though not the top student in the class, he held his own among an extremely gifted experienced group.''

\textbf{An example of hidden Language II}
\par\ \textbullet\ \textbf{A}: ``I do not doubt that she will go on to do first-rate research.''
\par\ \textbullet\ \textbf{B}: ``I have confidence in her ability.''

\textbf{An example of hidden Language II}
\par\ \textbullet\ \textbf{A}: ``She is one of the most talented students I've ever worked with.''
\par\ \textbullet\ \textbf{B}: ``She is the most enthusiastic student I've ever worked with.''

\newpage\textbf{Tips for recommenders}
\par\ \textbullet\ Approach potential letter writers at least several weeks in advance of the deadline;
\par\ \textbullet\ Choose your recommenders carefully;
\par\ \textbullet\ Take ``no'' for an answer;
\par\ \textbullet\ Avoid recommenders who ask you to draft your recommendation letter;
\par\ \textbullet\ Make life easy for your letter writer:
\par\quad\textopenbullet\ Provide them with your CV;
\par\quad\textopenbullet\ Offer to meet with them;
\par\quad\textopenbullet\ Give them clear and easy instructions on how to submit the letter;
\par\quad\textopenbullet\ Provide a link to information about the position or award.

\newpage\subsection{Write the personal statement}

\par Personal statements are used for a variety of purposes for admissions to medical or graduate school, for fellowships and scholarships, for internships and even sometimes for jobs.

\textbf{Tips for personal statements}
\par\ \textbullet\ Make it personal:
\par\quad\textopenbullet\ Speak from the heart;
\par\quad\textopenbullet\ Reveal who you are;
\par\quad\textopenbullet\ Strive for flair, not ``blah''.
\par\ \textbullet\ Give specific examples and stories
\par\quad\textopenbullet\ ``Show, don't tell'';
\par\ \textbullet\ Don't read your CV line by line
\par\quad\textopenbullet\ Highlight relevant experiences.
\par\ \textbullet\ Avoid big words you don't understand and avoid cliches;
\par\ \textbullet\ Show interest in/fatter your readers:
\par\quad\textopenbullet\ ``Do your homework'';
\par\quad\textopenbullet\ Be specific about why the specific program / institution / award appeals to you.
\par\ \textbullet\ Explain gaps and failures:
\par\quad\textopenbullet\ Don't ignore these in hopes that reviewers won't notice the issue!

\textbf{Elements: Opening / Lead}
\par\ \textbullet\ Start strong;
\par\ \textbullet\ Be creative;
\par\ \textbullet\ Be descriptive or tell a story;
\par\ \textbullet\ Impact who you are and what matters to you;
\par\ \textbullet\ Don't be boring!
\par\ \textbullet\ It's OK if it's a little longer if it's compelling.

\newpage\par\textbullet\ An example:
\par I was recently rereading an autobiography that I wrote in fourth grade. I had to list my favorite things to do.
My top four were running, solving puzzles, reading and writing. The foresight of children is amazing and then I actually go on to talk about that I was a competitive runner.
I solved puzzles through epidemiology and statistics, and I love to read, and then I finally get back to writing.
Not surprisingly, my passion for writing has also resurfaced. When I was a child, I did not dream of being a doctor or a scientist.
I dreamed of being a writer. I have been steered toward the hard sciences all my life. I have pondered careers in biochemistry, genetics and biostatistics.
Yet unfailingly, I find myself drawn back to my childhood whim. When I'm asked what I'm going to do when I finish my epidemiology PhD, I always answer laughingly.
``Actually, I'm going to be a writer and that's true.''

\textbf{Elements: Body of the essay}
\par\ \textbullet\ Where do you want to go?
\par\ \textbullet\ What experiences have led you to this point?
\par\ \textbullet\ What makes you a strong candidate?
\par\quad\textopenbullet\ Address weaknesses, and turn them into strengths.
\par\ \textbullet\ Why this specific program/institution/fellowship?

\textbf{Elements: Conclusion}
\par\ \textbullet\ End strong!
\par\ \textbullet\ Consider circling back to your opening story or description.

\newpage\quad

\newpage\section{Interview}
















































\end{document}
